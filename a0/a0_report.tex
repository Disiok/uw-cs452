\documentclass[11pt]{article}

\title{CS 452 Assignment 0: Polling Loop}
\author{
	Shun Da Suo\\
	20509411
}


\begin{document}
\maketitle

\noindent
This report contains the \textbf{user manual} and \textbf{design document} for a terminal program that controls the M\"arklin 6051 train set.

\section{User Manual}
\subsection{Getting the Executable}
\textbf{Option 1: Build from GitLab source}
\begin{enumerate}
	\item git clone gitlab@git.uwaterloo.ca:sdsuo/trains.git
	\item cd trains/a0
	\item make \&\& cp build/main.elf /u/cs452/tftp/ARM/sdsuo
\end{enumerate}
\textbf{Option 2: Use pre-built binary}
\begin{enumerate}
	\item cp /u9/sdsuo/cs452/a0/build/main.elf /u/cs452/tftp/ARM/sdsuo
\end{enumerate}

\subsection{Run the Program}
To ensure proper testing, please hard reset the train set before running the program.
Then issue the following commands in Redboot:
\begin{enumerate}
	\item load -b 0x00218000 -h 10.15.167.5 "ARM/sdsuo/main.elf"
	\item go
\end{enumerate}
Note that the program will initialize the train set upon startup (i.e. perform the following sequence of actions):
\begin{enumerate}
	\item Power on the train set
	\item Set all switches to straight position
	\item Initialize all trains to speed 0
	\item Enable sensor polling
\end{enumerate}


\subsection{Command Line Interface}
The command line interface displays:
\begin{enumerate}
	\item Top left: Current elapsed time (at 100 ms precision)
	\item Top right: Mean and max polling loop time
	\item Left: Available switches and their current states
	\item Right: Most recently activated sensors (up to 20)
	\item Bottom: Status, history, and command prompt
\end{enumerate}
Available commands include:
\begin{itemize}
	\item \textbf{go}: Power on the train set
	\item \textbf{stop}: Power off the train set
	\item \textbf{tr train{\_}number[1-80] train{\_}speed[0-14]}: Set speed for specified train
	\item \textbf{rv train\_number[1-80]}: Reverse specified train
	\item \textbf{sw switch\_number[0-255] switch\_direction[S/C]}: Set state for specified switch
	\item \textbf{swa switch\_direction[S/C]}: Set state for all available switches
	\item \textbf{ss}: Toggle sensor polling
\end{itemize}


\section{Design Document}
At a high level, the program: 
\begin{enumerate}
	\item keeps stateful structures on the main method's stack, 
	\item uses ring buffers to buffer UART communication,
	\item uses a polling loop for event handling.
\end{enumerate}

\subsection{Architecture}
Main components of the program are:
\begin{enumerate}
	\item [train.h] \textbf{TrainController}: Issuing train commands, scheduling delayed execution of commands
	\item [terminal.h] \begin{itemize}
		\item \textbf{SmartTerminal}: Issuing escape codes, rendering and updating user interface
		\item \textbf{TerminalController}: Parsing user input commands 
	\end{itemize}
	\item [time.h] \textbf{Clock}: Keeping time
	\item [track.h] \textbf{Track}: Storing information about track nodes, available switches and sensors
	\item [io.h] \textbf{BufferedChannel}: Providing read and write buffers for a UART communication channel 
	\item [ds.h] \textbf{RingBuffer}: Providing generic byte indexed ring buffer implementation
\end{enumerate}


\subsection{Polling Loop}
The program is driven by a main polling loop, which does the following:
\begin{enumerate}
	\item Poll Clock: \begin{itemize}
		\item Check timer's value register. If the current value is larger than the previously recorded value, it means that the timer underflowed and was reset. Thus increment the clock by the tick period (i.e. 10 ms).
		\item Check current time against display update period (i.e. 100 ms). Notify SmartTerminal to update display as necessary. 
	\end{itemize}
	\item Poll BufferedChannel for terminal: \begin{itemize}
		\item If ready to receive, read one byte onto read ring buffer.
		\item If ready to transmit and write ring buffer is not empty, move one byte to terminal UART.
	\end{itemize} 
	\item Poll BufferedChannel for train set: \begin{itemize}
		\item If ready to receive, read one byte onto read ring buffer.
		\item If ready to transmit and write ring buffer is not empty, move one byte to train set UART.
	\end{itemize}
	\item Poll TrainController: If display time has changed (i.e. 100ms has elapsed since the previous check) \begin{itemize}
		\item Check and execute delayed command ring buffers for train set commands scheduled
		\item Check read ring buffer for sensor update bytes. Parse raw bytes into sensor ids and update sensor ring buffer. Notify SmartTerminal to update display.
		\item If read ring buffer is empty and sensor polling is enabled, send commands to request sensor updates.
	\end{itemize}
	\item Poll TerminalController: If read ring buffer is not empty \begin{itemize}
		\item And visible character is read, echo input to cursor position of prompt display, and append to command buffer.
		\item And enter is read, call TrainController with completed command and update history display with command.
		\item And backspace is read, clear input at cursor position of prompt display and remove last character from command buffer.
	\end{itemize}
\end{enumerate}

\subsection{Implementation Details}
\subsubsection{Time Keeping}
The 32-bit timer is used for time keeping. More specifically, we use the lower clock speed at 2kHz, and set the mode to periodic. At this clock speed, we can track time at the granularity of 0.5 ms, which is sufficient for our needs. The initial load value is set to 20 for a timer period of 10 ms (i.e. every underflow/reset corresponds to one period of 10 ms).\\

\noindent
\textbf{Response to Question 1:} How do you know that your clock does not miss updates or lose
time?) \\

\noindent
This implies that as long as the polling loop runs within 10 ms, we will not lose any periodic resets. Empirically, we observe max polling loop time to be well below this threshold.

\subsubsection{Ring Buffer}
Ring buffers are the only non-primitive data structures used in the program. They are used extensively for decoupling producers and consumers of messages. Here, ring buffers are implemented with stack allocated static arrays, with pointers to both head (i.e. index of first byte) and tail (i.e. index of last byte + 1). They operate first in first out (FIFO). They are currently configured to silently overwrite the oldest bytes when the max size is reached.

\subsubsection{Delayed Commands}
To execute commands for reversing trains and flipping switches correctly, we use two ring buffers to store scheduled delayed commands. More specifically, we use a 5-byte protocol (i.e. {a [4-byte time ms, 1-byte command]} pair). When polling the TrainController, we check if the delayed command buffers are empty, and peak the timestamp at head (i.e. oldest delayed command). While the timestamp is less than the current time, we shrink the delayed command buffers and send the 1-byte command to the train controller.

\subsubsection{Sensor Polling}
To poll for track sensor updates. We periodically issue the one byte command to read all 5 sensor modules. To account for delays in train set response and transmission, we set this period to 100 ms. To further make sure we can keep up with the sensor updates, we do not issue new command to request sensor update until we have read off all previous bytes from the read buffer.\\

\noindent
The 10-byte sensor update response is parsed together. We calculate and push the sensor ids onto the sensor update ring buffer. It is okay to directly reuse the byte indexed ring buffer, since the range for the sensor ids is 0 to 255. We do, however, set an additional size limit of 20 for the sensor update ring buffer, since we only care about the latest 20 sensor hits.\\

\noindent
\textbf{Response to Question 2:} How long does the train hardware take to reply to a sensor query? \\

\noindent
To verify the delays in train set response and transmission, we use a minimal polling loop that only polls for BufferedChannel updates and Clock updates. From the time the sensor update command is issued, it takes approximately \textbf{18 ms} to receive the first byte of response. At the rate of 2.4Kb/s (i.e. ~0.24KB/s), it takes approximately 4.2 ms to transmit each byte over the communication channel. Thus, it takes approximately \textbf{9.6 ms} ($=18ms-4.2ms*2$) for the train set to query all 5 sensor modules and reply with the update. It takes approximately \textbf{60 ms} to receive all ten bytes, which is roughly what we expect ($18ms + 4.2ms *9 \simeq 56ms$). 

\section{Source Code}

\textbf{Top level}\\
f984e0cd6d3fbd4aa078f0d0dbf5aea9  ./Makefile\\
9bcd562566ba01c869c8c2759ff90e64  ./orex.ld\\

\noindent
\textbf{Headers}\\
a7895ea1ba118b0b4c7f43c80b517ad7  ./include/bwio.h\\
9af226f127c1fd759530cd45236c37b8  ./include/ts7200.h\\
e3bd55ddffe9b1da2d05cd1000579b88  ./include/io.h\\
f78e35718b07f5b09e1d593c5a6f4a27  ./include/terminal.h\\
45c696fe33aa8211ffb49de54290c028  ./include/time.h\\
1cd15a9deef660d88bc2418ce67b0f52  ./include/ds.h\\
1352f3743944badbb8c2399e6fb2ccd4  ./include/track\_data.h\\
e64108099d229b1cc61e739073ba5bf8  ./include/train.h\\
0955a30671db0a0150c7a03369254f30  ./include/util.h\\
8ffa85dd374f6f1226bee787fa6bf71d  ./include/track\_node.h\\
9c766bc8610c0b07c034520fd2e252b9  ./include/track.h\\

\noindent
\textbf{Implementations}\\
2d93650ecdab0ab0f355d5cd0fce5067  ./src/bwio.c\\
b74c67b2873d2b8002ac9eeb473fb80a  ./src/ds.c\\
d11cd29f4d84f43b35e0233b06a013cb  ./src/io.c\\
e1ae9c610bcdbbd6fb2c476d3ffeeea3  ./src/io.s\\
532f6eecd9a0a1ff7b41dce7bc0dd673  ./src/main.c\\
0583f58d141f1f64917956f04a0b45a6  ./src/terminal.c\\
0522d6cea2a6ca669044252b3480c6cd  ./src/time.c\\
5d8c49a0fd5ba8ba9902f95db6389a72  ./src/train.c\\
beeff5f6addf3f45eae797fc905d4816  ./src/track\_data.c\\
d0dca3ac2c548f64339f3c726d96b4c7  ./src/util.c\\
b2a36b4b7803fc5705e584650c21b60a  ./src/track.c\\


\end{document}
